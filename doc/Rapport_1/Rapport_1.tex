\documentclass[conference]{IEEEtran}
\IEEEoverridecommandlockouts
% The preceding line is only needed to identify funding in the first footnote. If that is unneeded, please comment it out.
\usepackage{cite}
\usepackage[utf8]{inputenc}
\usepackage{amsmath,amssymb,amsfonts}
\usepackage{algorithmic}
\usepackage{graphicx}
\usepackage{textcomp}
\usepackage{xcolor}
\def\BibTeX{{\rm B\kern-.05em{\sc i\kern-.025em b}\kern-.08em
    T\kern-.1667em\lower.7ex\hbox{E}\kern-.125emX}}
\begin{document}

\title{Git : Galileo's Internet of Things
}

\author{\IEEEauthorblockN{JETON Alex, SIMONNET Adrien, WU Pierre}
\IEEEauthorblockA{\textit{Cours: Internet des Objets} \\
\textit{Professeur: OSMANI Aomar}\\
DATE: 3 Novembre 2019\\
}

}

\maketitle

\section{Introduction}
    Dans le cadre d'une licence informatique, nous sommes chargés de réaliser un objet connecté. L'Internet des Objets Connectés (Iot) désigne des objets physiques capables d'émettre des données grâce à des capteurs (objet connecté), le réseau par lequel ces données transitent, les plateformes capables des les recueillir et de les analyser. Étant étudiants, nos ressources sont limitées, le projet dépend des moyens fournit par l'université.
    
\section{Le projet}
\subsection{Objectif}
L'objectif est de concevoir un engin motorisé, électrique et télécommandé.\newline 
La description technique est :
\begin{itemize}
    \item Une batterie alimentera les deux moteurs.
    \item Trois roues seront disposées en triangle dont deux motrices.
    \item Pour avancer, les moteurs sont à la même puissance.
    \item Pour tourner à droite (resp. gauche), il suffit de réduire la puissance du moteur droit (resp. gauche).
    \item  Un accéléromètre sera embarqué afin de déterminer l'évolution de sa position et ainsi déterminer sa vitesse.
    \item La vitesse pourra être affichée à l'aide d'un affichage numérique connecté à la voiture.
    \item Pour rendre cette voiture quasi-autonome, des capteurs de présence y seront ajoutés.
\end{itemize}

Par la suite, l'engin peut devenir une voiture. Mais, nous devons respecter les contraintes de poids et de puissance.
\subsection{Les moyens mis en oeuvre}
L'ensemble du matériel électronique nécessaire à la conception du produit est déjà disponible dans le Starter Kit Arduino. Il nous reste à nous procurer et fabriquer la carcasse, c'est à dire un support sur lequel poser l'électronique, ainsi que les roues. Pour le software, nous utiliserons Arduino comme langage de programmation et Arduino IDE comme environnement de développement intégré.

\subsection{Dates}
Nous devons respecter certaines contraintes de temps. Elles sont définies par le tableau suivant.
\\ \\
\begin{tabular}{|*{3}{c|}}
  \hline
  Activités & Date de début & Date de fin \\
  \hline
  Début du projet & 16/10/2019 &  \\
  1er rapport & 1/11/2019 & 3/11/2019 \\
  1ère présentation & 4/11/2019 &   \\
  2ème rapport & 4/11/2019 & 10/11/2019  \\
  2ème présentation & 13/11/2019 &   \\
  3ème présentation & 14/11/2019 & 15/12/2019  \\
  Présentation finale & 18/11/2019 &  \\
  \hline
\end{tabular}

\subsection{L'état du marché}
L'Internet des objets est en expansion. Il s'agit de la 3ème évolution d'internet. Ainsi, les entreprises s'intéressent aux objects connectées (Roborock, LG, Samsung). Les engins motorisés n'échappent pas à cette loi du marché ; Certains produits avec une ressemblance similaire au nôtre commencent à surgir. Nous pouvons citer : Les robots aspirateurs et les robots poubelles. 

\subsection{Les normes en vigueur}
Nous utiliserons uniquement les matériaux fournis par le kit Starter "Arduino". Il faut uniquement s'assurer que la manière dont on utilisera le matériel respectera toujours les normes en vigueur.

\section{Conclusion}
Ce projet, risque d'être formateur pour l'ensemble de l'équipe. Nous acquierons des compétences dans le domaine de la programmation, de la robotique et de la gestion de projet. De plus, il donne un aperçu de la vie en entreprise ; Nous préparant au stage du semestre 6.\\


\end{document}
